% Options for packages loaded elsewhere
\PassOptionsToPackage{unicode}{hyperref}
\PassOptionsToPackage{hyphens}{url}
%
\documentclass[
]{article}
\title{Overivew}
\author{Dr.~Eric Asare}
\date{29/11/2021}

\usepackage{amsmath,amssymb}
\usepackage{lmodern}
\usepackage{iftex}
\ifPDFTeX
  \usepackage[T1]{fontenc}
  \usepackage[utf8]{inputenc}
  \usepackage{textcomp} % provide euro and other symbols
\else % if luatex or xetex
  \usepackage{unicode-math}
  \defaultfontfeatures{Scale=MatchLowercase}
  \defaultfontfeatures[\rmfamily]{Ligatures=TeX,Scale=1}
\fi
% Use upquote if available, for straight quotes in verbatim environments
\IfFileExists{upquote.sty}{\usepackage{upquote}}{}
\IfFileExists{microtype.sty}{% use microtype if available
  \usepackage[]{microtype}
  \UseMicrotypeSet[protrusion]{basicmath} % disable protrusion for tt fonts
}{}
\makeatletter
\@ifundefined{KOMAClassName}{% if non-KOMA class
  \IfFileExists{parskip.sty}{%
    \usepackage{parskip}
  }{% else
    \setlength{\parindent}{0pt}
    \setlength{\parskip}{6pt plus 2pt minus 1pt}}
}{% if KOMA class
  \KOMAoptions{parskip=half}}
\makeatother
\usepackage{xcolor}
\IfFileExists{xurl.sty}{\usepackage{xurl}}{} % add URL line breaks if available
\IfFileExists{bookmark.sty}{\usepackage{bookmark}}{\usepackage{hyperref}}
\hypersetup{
  pdftitle={Overivew},
  pdfauthor={Dr.~Eric Asare},
  hidelinks,
  pdfcreator={LaTeX via pandoc}}
\urlstyle{same} % disable monospaced font for URLs
\usepackage[margin=1in]{geometry}
\usepackage{graphicx}
\makeatletter
\def\maxwidth{\ifdim\Gin@nat@width>\linewidth\linewidth\else\Gin@nat@width\fi}
\def\maxheight{\ifdim\Gin@nat@height>\textheight\textheight\else\Gin@nat@height\fi}
\makeatother
% Scale images if necessary, so that they will not overflow the page
% margins by default, and it is still possible to overwrite the defaults
% using explicit options in \includegraphics[width, height, ...]{}
\setkeys{Gin}{width=\maxwidth,height=\maxheight,keepaspectratio}
% Set default figure placement to htbp
\makeatletter
\def\fps@figure{htbp}
\makeatother
\setlength{\emergencystretch}{3em} % prevent overfull lines
\providecommand{\tightlist}{%
  \setlength{\itemsep}{0pt}\setlength{\parskip}{0pt}}
\setcounter{secnumdepth}{-\maxdimen} % remove section numbering
\ifLuaTeX
  \usepackage{selnolig}  % disable illegal ligatures
\fi

\begin{document}
\maketitle

\hypertarget{overview-section}{%
\section{\texorpdfstring{ 1.0. Overview Section
}{ 1.0. Overview Section }}\label{overview-section}}

\textless font size = ``5'', color=`black'\textgreater{} You are here
because you clicked on the Overview icon in the App shown in Figure 1
below:

\hypertarget{figure-1.-overview-section}{%
\subsubsection{\texorpdfstring{ Figure 1. Overview Section
}{ Figure 1. Overview Section }}\label{figure-1.-overview-section}}

\includegraphics{overview1.png}

\textless font size = ``5'', color=`black'\textgreater{} This section
summarizes the structure of the App, including how to navigate it. The
general objective of the App is to estimate the net-present benefit
(NPB) of wetland drainage at the field level and at the watershed or
landscape level; it also assesses the effect of changes in crop prices
on the NPB. Full details on how we calculated NPB and its underlying
assumptions are provided in a User-guide which can be downloaded using
the download link to the left of this section. We describe how to
navigate the field-level and landscape level sections below. In each
section we state the specific outputs and the inputs users can change
(user-defined parameters) to change the outputs.

\hypertarget{field-level-section}{%
\section{\texorpdfstring{ 1.2. Field Level Section
}{ 1.2. Field Level Section }}\label{field-level-section}}

\textless font size = ``5'', color=`black'\textgreater{} Figure 2 shows
the Field-level section if click the icon circled in red. You will find
two main sections which are the user-defined and output section.

\hypertarget{figure-2.-layout-of-the-field-level-section}{%
\subsubsection{\texorpdfstring{ Figure 2. Layout of the Field-level
Section
}{ Figure 2. Layout of the Field-level Section }}\label{figure-2.-layout-of-the-field-level-section}}

\includegraphics{fieldlevel.png}

\textless font size = ``5'', color=`black'\textgreater{} The output
section shows the results of net-present benefit of wetland drainage
analysis at the field level. The specific outputs are:

\begin{itemize}
\tightlist
\item
  Plot of present value of cultivated crops on drained wetland area.
\item
  Net-present benefit of wetland drainage.\\
\item
  Net-present benefit of wetland drainage given percentage change in
  crop prices.
\item
  Annualized Net-present benefit of wetland drainage.\\
\item
  Annualized Net-present benefit of wetland drainage given percentage
  change in crop prices.
\end{itemize}

\textless font size = ``5'', color=`black'\textgreater{} There are
variables that are used to produce the outputs above. For instance the
outputs in figure 2 are generated with default parameters. Users are
able to change the default parameters with user-defined values which are
grouped by the 3 grey circles numbered (1) to (3). The grey circles are
described below:

\begin{itemize}
\tightlist
\item
  grey circle 1: by clicking on this circle, the user is able to change
  the levels of wetland area (acres), wetland location (at field margin
  or not), efficiency gain (cost savings from eliminating obstructions
  to farm-machinery operations due to wetlands), delayed seeding,
  discount rate and planning horizon.\\
\item
  grey circle 2: by clicking on this circle, the user is able to select
  the soil zone of the field and cultivated crops to edit crop
  production information.\\
\item
  grey circle 3: by clicking on this, the user is able to choose the
  expected percentage change in crop prices and the direction of change.
\end{itemize}

\hypertarget{landscape-level-section}{%
\section{\texorpdfstring{ 1.3. Landscape Level Section
}{ 1.3. Landscape Level Section }}\label{landscape-level-section}}

\textless font size = ``5'', color=`black'\textgreater{} Figure 3 shows
the landscape-level section if click the icon circled in red. You will
find two main sections which are the user-defined and output section.

\hypertarget{figure-3.-layout-of-the-landscape-level-section}{%
\subsubsection{\texorpdfstring{ Figure 3. Layout of the Landscape-level
Section
}{ Figure 3. Layout of the Landscape-level Section }}\label{figure-3.-layout-of-the-landscape-level-section}}

\#\includegraphics{landscape.png}

\textless font size = ``5'', color=`black'\textgreater{} The output
section shows the results of net-present benefit of wetland drainage
analysis at the watershed or landscape level. The specific outputs are:

\begin{itemize}
\tightlist
\item
  Wetland drainage supply curve: which shows wetland areas (acres/year)
  that could be drained for different levels of annualized net-present
  benefit (ANPB) of drainage.
\item
  Annualized net-present benefit of wetland drainage summary table: in
  the summary table which you can find by clicking on the grey circle
  below wetland drainage supply curve, we have estimates of:

  \begin{enumerate}
  \def\labelenumi{(\arabic{enumi})}
  \tightlist
  \item
    percentage of quarter-sections(1 QS = 160 acres) where ANPB
    \textgreater{} 0; sum of wetland acres in QS where ANPB
    \textgreater0; wetland acres conserved per year given wetland
    conservation budget.
  \item
    Changes in estimates in (1) for a percentage change in crop prices.
  \end{enumerate}
\end{itemize}

\textless font size = ``5'', color=`black'\textgreater{} Users can
specify their user-defined inputs by clicking on the grey circles number
(1) to (5). The grey circles are described below:

\begin{itemize}
\tightlist
\item
  grey circle 1: by clicking on this circle, the user is able to view
  and download the required data for landscape analysis. The user can
  change the values in the data and upload it in the App.\\
\item
  grey circle 2: by clicking on this circle, the user can upload the
  edited data.\\
\item
  grey circle 3: by clicking on this, the user is able to choose the
  expected percentage change the levels of levels of number of
  quarter-sections, wetland area (acres), wetland location (at field
  margin or not), efficiency gain (cost savings from eliminating
  obstructions to farm-machinery operations due to wetlands), delayed
  seeding, proportion of soil zones, discount rate and planning horizon.
\item
  grey circle 4: by clicking on this circle, the user is able to provide
  the level of wetland conservation budget.\\
\item
  grey circle 5: by clicking on this, the user is able to choose the
  expected percentage change in crop prices and the direction of change.
\end{itemize}

\end{document}
